\documentclass[a4paper, 11pt]{article}

\usepackage{hyperref}
\usepackage{listings}
\usepackage{MnSymbol}

%use arrows to indicate line breaks in listings
\lstset{prebreak=\raisebox{0ex}[0ex][0ex]
        {\ensuremath{\rhookswarrow}}}
\lstset{postbreak=\raisebox{0ex}[0ex][0ex]
        {\ensuremath{\rcurvearrowse\space}}}
\lstset{breaklines=true, breakatwhitespace=true}
\lstset{numbers=left, numberstyle=\scriptsize}


%define general listing command
\newcommand{\includecode}[1]{
    \lstinputlisting[   language=C++,
                        basicstyle=\small,
                        breaklines=true,
                        breakautoindent=true,
                        linewidth=15cm]{#1}}

\begin{document}
    \begin{tabular}{ l  c  c  c  r  r}
        topic & problem & name & author & location\\
      \hline
        ACM & \textit{Given a set of intervals...} & Aliens & Ben & \ref{sec:aliens} \\
        & \textit{Given ...} & Checking Change & Ben & \ref{sec:checking_change} \\
    \end{tabular}

    \begin{section}{Aliens}
        \label{sec:aliens}
        \includecode{../problems/Aliens/Aliens1.cpp}
    \end{section}
    \begin{section}{Checking Change}
        \label{sec:checking_change}
        \includecode{../problems/Checking_Change/CheckingChangeBen.cpp}
    \end{section}

\end{document}
